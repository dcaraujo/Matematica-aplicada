%%% Modelo de Lista de Exercícios para Universidade
%%% Criado por Diogo Roberto R. Freitas (diogo@poli.br)
%%% Livre para alterações
\documentclass[12pt,onepage,a4paper]{memoir}

%% Language and font encodings
\usepackage[english,portuges]{babel}
\usepackage[T1]{fontenc}
\usepackage[utf8]{inputenc}

%% Sets page size and margins
\usepackage[a4paper,top=2.5cm,bottom=2cm,left=2cm,right=2cm,marginparwidth=1.75cm]{geometry}
\setlength\parindent{0cm} % Tamanho da tabulação dos parágrafos

%%% ATENÇÃO!!! %%%
%%% Preencha estes comandos com suas informações
\newcommand{\logo}{\includegraphics[width=0.4\textwidth]{fig/unichristus}} % inclua o arquivo com o logo da instituição
\newcommand{\univ}{Centro Universitário Christus}
\newcommand{\escola}{Sistemas de Informação}
\newcommand{\disc}{Matemática Aplicada}
\newcommand{\auth}{Prof. Daniel Araújo}
\newcommand{\email}{\url{repositoriouni@gmail.br}}
\newcommand{\sitedisc}{\url{--}} % site da disciplina
\newcommand{\cabec}{UNICHRISTUS} % cabeçalho a partir da 2ª página
\newcommand{\tit}{Resolução da NP1}
\newcommand{\tp}[1]{(Tópico: #1)- }
%%%%%%%%%

%% Useful packages
\usepackage{amsmath,amsfonts}
\usepackage{graphicx}
\usepackage[colorinlistoftodos]{todonotes}
\usepackage[colorlinks=true, allcolors=blue]{hyperref}
\linespread{1.25}
\usepackage{graphicx}
\usepackage{nicefrac}
\usepackage[tight]{units}
\usepackage[justification=centering]{caption}
\usepackage{subcaption}
\usepackage{lastpage}
\usepackage{pstricks}
\usepackage{url}% ou hyperref
%\usepackage{breakurl}
\usepackage{multirow}
\usepackage{tabulary}
\usepackage{longtable}
\usepackage{microtype}% improves the spacing between words and letters
\usepackage{booktabs}%  helps you improve the quality of your LaTeX tables
\usepackage{rotfloat}
\usepackage{rotating}

%%% NÃO ALTERAR ESTES COMANDOS 
%%% Configura a primeira página
\makepagestyle{1pagina}
\makeoddhead{1pagina}{
	\logo \\
    \vspace{5pt}
    %\textsf{\univ \\ \escola \\} %%% Caso o logo não tenha o nome da universidade descomente essa linha
    \textsf{Disciplina: \disc \\
	\auth~(\email) \\
    \sitedisc }
    }{}{}
\makeoddfoot{1pagina}{\tiny \cabec}{}{\scriptsize Página \thepage~de \thelastpage}
\makefootrule{1pagina}{\textwidth}{\normalrulethickness}{5pt}
%%% Configura as demais páginas
\makepagestyle{paginacomum}
\makeevenhead{paginacomum}{\textsf{\scriptsize \cabec~-- \auth}}{}{}
\makeevenfoot{paginacomum}{\tiny \cabec}{}{\scriptsize Página \thepage~de \thelastpage}
\makeoddhead{paginacomum}{\textsf{\scriptsize \cabec~-- \auth}}{}{}
\makeoddfoot{paginacomum}{\tiny \cabec}{}{\scriptsize \thepage~de \thelastpage}
\makeheadrule{paginacomum}{\textwidth}{\normalrulethickness}
\makefootrule{paginacomum}{\textwidth}{\normalrulethickness}{5pt}
\pagestyle{paginacomum}
%%%


%%% Início do documento
\begin{document}
\thispagestyle{1pagina}
\vspace*{2.5cm} %%% Caso o cabeçalho cubra estes texto aumente o "vspace"
\fbox{
  \begin{minipage}{\textwidth}
    Instruções: Resolução da NP1
    \end{minipage}
  
}

\vspace{0.5cm}
\textbf{\textsf{\large \tit}} %%% não altere aqui



\begin{enumerate} % início das questões, pode escrever aqui
  % Quest 1

 \item Considere os polinômios $f(x) = ax - 6$ e $g(x) = x^2 - x + 10$. Qual deve ser o valor de “$a$” para que
          $f(x) - g(x)$ seja um produto notável.

          Resultado:

       \begin{align*}
       f(x) &= g(x) \\
        x^2 - x + 10   & = ax - 6  \\
         x^2 -(1+a)x + 16 & = 0
      \end{align*}
      
      Se assumirmos que $x^2 -(1+a)x + 16 = x^2 -8x + 16$.

        \begin{align*}
          1+a &= 8 \\
          a & = 7
        \end{align*}
      


          \item  Dois indivíduos tentam se comunicar utilizando expressões matemáticas. A transmissão de um número é obtida resolvendo uma expressão matemática. O indivíduo “A” envia a seguinte mensagem 

\begin{equation}
  \left( \frac{x + \sqrt{x^2-4x}}{x - \sqrt{x^2-4x}}  - \frac{x - \sqrt{x^2-4x}}{x + \sqrt{x^2-4x}} \right)\left( \frac{1}{\sqrt{x^2-4x}}\right)
\end{equation}
Você é o indivíduo B e precisa simplificar expressão para determinar que número
dos conjuntos reais a mensagem se refere.

Resposta

\begin{align}
  \left( \frac{x + \sqrt{x^2-4x}}{x - \sqrt{x^2-4x}}  - \frac{x - \sqrt{x^2-4x}}{x + \sqrt{x^2-4x}} \right)\left( \frac{1}{\sqrt{x^2-4x}}\right)& \nonumber \\
  \left( \frac{\left(x + \sqrt{x^2-4x}\right)^2}{x^2 - x^2+4x}  - \frac{\left(x - \sqrt{x^2-4x}\right)^2}{x^2  - x^2+4x}\right)\left( \frac{1}{\sqrt{x^2-4x}}\right)& \nonumber \\
  \left( \frac{\left(x + \sqrt{x^2-4x}\right)^2 - \left(x - \sqrt{x^2-4x}\right)^2}{4x}\right)\left( \frac{1}{\sqrt{x^2-4x}}\right)& \nonumber \\
  \left( \frac{x^2  + 2x\sqrt{x^2-4x} + x^2-4x - \left(x^2 - 2x\sqrt{x^2-4x} + x^2-4x\right)}{4x}\right)\left( \frac{1}{\sqrt{x^2-4x}}\right)& \nonumber \\
  \left( \frac{4x\sqrt{x^2-4x}}{4x}\right) \left( \frac{1}{\sqrt{x^2-4x}}\right) & \nonumber \\
   \left( \frac{4x}{4x}\right) \left( \frac{\sqrt{x^2-4x}}{\sqrt{x^2-4x}}\right)  = 1
\end{align}


\item Resolva os itens a seguir

\begin{enumerate}
\item Obtenha a raiz cúbica 3375

  Resposta:  $\sqrt[3]{3^35^3} = 15$
 
\item Calcule $\sqrt[1/3]{\left( \sqrt[4]{16} \right)^{2/3}}$

  Resposta :  $\sqrt[1/3]{2^{2/3}} = 4 $

\item  Simplifique $\sqrt{\frac{a}{\sqrt[3]{a}}}$

  Resposta :  $\sqrt[3]{a}$
 
\end{enumerate}


\item Seja $C=\{ x \in \mathbb{R} | -6 \leq x \leq 6  \}$, responda os itens a seguir.  

    \begin{enumerate}[a)]
    \item  Qual sua representação através da reta?
    \item  Se $y = \sqrt{-x}$ , represente o conjunto ao qual y pertence.
    \item Se $y = -\sqrt{x}$, represente o conjunto ao qual y pertence.
    \end{enumerate}

Resposta

       \begin{enumerate}[a)]
       \item  Qual sua representação através da reta?


         \begin{figure}[h]
           \centering
           \begin{tikzpicture}
             \draw [->] (-3,0) -- (3,0) node[below] {$x$};
             \draw [red] (-2,0) node[circle,fill=black!90](pointOne) {} -- (2,0) node[circle,fill=black!90](pointTwo) {};
             \node[below] at (pointTwo) {$6$};
             \node[below] at (pointOne) {$-6$};
           \end{tikzpicture}
           \caption{Representação na reta}
         \end{figure}


       \item  Se $y = -\frac{1}{\sqrt{x}}$, represente o conjunto ao qual y pertence.

         Resposta: $C_y = \{y \in \mathbb{R} |  -\frac{1}{\sqrt{6}} \leq x <
         0\}$  e domínio $C_x=\{ x \in \mathbb{R} | 0 < x \leq 6  \}$
         
       \item  Se $y = \sqrt{-x}$ , represente o conjunto ao qual y pertence.

Resposta: $C_y = \{y \in \mathbb{R} |  0 \leq x \leq \sqrt{6}\}$ e domínio $C_x=\{ x \in \mathbb{R} | -6 \leq x \leq 0  \}$        

\end{enumerate}



\item Determine a forma fatorada do produto $(x^2 - 14x + 49 )(x^2 + 14x + 49 )$

  Resposta:

  \begin{align}
    (x-7 )^2(x+7)^2
  \end{align}
  
\item Determine a for-ma simplificada $(x^2 + 14x + 49)( x^2 - 49)/(x^2 - 14x +  49)$
   
  Resposta:

  \begin{align}
    \frac{(x+7)^2(x-7)(x+7)}{(x-7)^2} \nonumber \\
    \frac{(x+7)^2(x+7)}{(x-7)}  \nonumber \\
    \frac{(x+7)^3}{(x-7)}
  \end{align}
\item  A forma simplificada da razão entre os polinômios $x^3 - 8y^3$ e $x^2 + 2xy + 4y^2$

  Resposta :

  \begin{align}
    \frac{(x-2y)(x^2 + 2xy + 4y^2)}{x^2 + 2xy + 4y^2} \nonumber \\
     x-2y
  \end{align}

 
  
\end{enumerate} % fim das questões
\end{document}