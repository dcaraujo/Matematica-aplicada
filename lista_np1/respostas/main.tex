%%% Modelo de Lista de Exercícios para Universidade
%%% Criado por Diogo Roberto R. Freitas (diogo@poli.br)
%%% Livre para alterações
\documentclass[12pt,onepage,a4paper]{memoir}

%% Language and font encodings
\usepackage[english,portuges]{babel}
\usepackage[T1]{fontenc}
\usepackage[utf8]{inputenc}

%% Sets page size and margins
\usepackage[a4paper,top=2.5cm,bottom=2cm,left=2cm,right=2cm,marginparwidth=1.75cm]{geometry}
\setlength\parindent{0cm} % Tamanho da tabulação dos parágrafos

%%% ATENÇÃO!!! %%%
%%% Preencha estes comandos com suas informações
\newcommand{\logo}{\includegraphics[width=0.4\textwidth]{fig/unichristus}} % inclua o arquivo com o logo da instituição
\newcommand{\univ}{Centro Universitário Christus}
\newcommand{\escola}{Sistemas de Informação}
\newcommand{\disc}{Matemática Aplicada}
\newcommand{\auth}{Prof. Daniel Araújo}
\newcommand{\email}{\url{repositoriouni@gmail.br}}
\newcommand{\sitedisc}{\url{--}} % site da disciplina
\newcommand{\cabec}{UNICHRISTUS} % cabeçalho a partir da 2ª página
\newcommand{\tit}{Radiciação e Potência}
\newcommand{\tp}[1]{(Tópico: #1)- }
%%%%%%%%%

%% Useful packages
\usepackage{amsmath,amsfonts}
\usepackage{graphicx}
\usepackage[colorinlistoftodos]{todonotes}
\usepackage[colorlinks=true, allcolors=blue]{hyperref}
\linespread{1.25}
\usepackage{graphicx}
\usepackage{nicefrac}
\usepackage[tight]{units}
\usepackage[justification=centering]{caption}
\usepackage{subcaption}
\usepackage{lastpage}
\usepackage{pstricks}
\usepackage{url}% ou hyperref
%\usepackage{breakurl}
\usepackage{multirow}
\usepackage{tabulary}
\usepackage{longtable}
\usepackage{microtype}% improves the spacing between words and letters
\usepackage{booktabs}%  helps you improve the quality of your LaTeX tables
\usepackage{rotfloat}
\usepackage{rotating}

%%% NÃO ALTERAR ESTES COMANDOS 
%%% Configura a primeira página
\makepagestyle{1pagina}
\makeoddhead{1pagina}{
	\logo \\
    \vspace{5pt}
    %\textsf{\univ \\ \escola \\} %%% Caso o logo não tenha o nome da universidade descomente essa linha
    \textsf{Disciplina: \disc \\
	\auth~(\email) \\
    \sitedisc }
    }{}{}
\makeoddfoot{1pagina}{\tiny \cabec}{}{\scriptsize Página \thepage~de \thelastpage}
\makefootrule{1pagina}{\textwidth}{\normalrulethickness}{5pt}
%%% Configura as demais páginas
\makepagestyle{paginacomum}
\makeevenhead{paginacomum}{\textsf{\scriptsize \cabec~-- \auth}}{}{}
\makeevenfoot{paginacomum}{\tiny \cabec}{}{\scriptsize Página \thepage~de \thelastpage}
\makeoddhead{paginacomum}{\textsf{\scriptsize \cabec~-- \auth}}{}{}
\makeoddfoot{paginacomum}{\tiny \cabec}{}{\scriptsize \thepage~de \thelastpage}
\makeheadrule{paginacomum}{\textwidth}{\normalrulethickness}
\makefootrule{paginacomum}{\textwidth}{\normalrulethickness}{5pt}
\pagestyle{paginacomum}
%%%


%%% Início do documento
\begin{document}
\thispagestyle{1pagina}
\vspace*{2.5cm} %%% Caso o cabeçalho cubra estes texto aumente o "vspace"
\fbox{
  \begin{minipage}{\textwidth}
    Instruções: Exercícios de revisão para  NP1.
    \end{minipage}
  
}

\vspace{0.5cm}
\textbf{\textsf{\large \tit}} %%% não altere aqui



\begin{enumerate} % início das questões, pode escrever aqui
  % Quest 1
\item Resolva a expressão $\sqrt[3]{2\left( \sqrt[2]{9}+2\sqrt[2]{25} - 1
    \right)}$

  Resposta:
  \begin{align*}
    f(x) & = \sqrt[3]{2\left( \sqrt[2]{9}+2\sqrt[2]{25} - 1
           \right)} \\
         & = \sqrt[3]{2\left( 3 + 10 - 1\right)} \\
         & = \sqrt[3]{24} \\
         & = 2\sqrt[3]{3}           
  \end{align*}

\item Obtenha a raiz cúbica 3375

  Resposta:  $\sqrt[3]{3^35^3} = 15$
 
\item Calcule $\sqrt[1/3]{5^{2/3}}$

  Resposta :  $\sqrt[1/3]{5^{2/3}} = 25$

  
\item Calcule $9^{\frac{3}{2}} + 32^{0.8}$

  Resposta

  \begin{align*}
    f(x) & = 9^{\frac{3}{2}} + 32^{0.8} \\
         & = 3^{3}    + 2^{5\frac{8}{10}} \\
         & = 27  +  16 \\
         & = 43           
  \end{align*}
  
  \item Mariana tinha 121 balas ela prometeu dar a raiz quadrada de suas balas a
    seu primo Igor. Depois de dar as balas para seu primo, deu 27 balas a sua
    irmã mais nova. Com quantas balas ficou Mariana?

    Resposta:

    \begin{itemize}
    \item   $x = \sqrt{121} = 11$  é quantidade de balas dadas ao seu primo
      \item $y = 121 - x  - 27 = 83$ é quantidade de balas que mariana ficou
      \end{itemize}


      
    \item Carlos foi desafiado pelo seu amigo: descobrir o número cujo dobro,
      subtraindo-se 2 resulte na  raiz quadrada de 144. Qual é esse
      número?

      Resposta:   

      \begin{align*}
        2x - 2 &= \sqrt{144} \\
        x & = \frac{12-2}{2} \\
        x & = 5
      \end{align*}
      

      \item Se elevarmos um número natural ao quadrado e tirarmos a raiz
        quadrada do resultado da potência. O que acontecerá?


        Resultado:

        \begin{align*}
          \sqrt{x^2} = |x|
        \end{align*}
       
        \item Considere os polinômios $f(x) = ax -2$ e $g(x) = x^2 - x + 7$. Se
          ambos forem iguais em $x=3$, qual deve ser o valor de “$a$” para que
          isso seja verdade

          Resultado:


       \begin{align*}
       f(x) &= g(x) \\
        x^2 - x + 7   & = ax - 2  \\
         x^2 -(1+a)x + 9 & = 0
      \end{align*}
      
      Para que $3$ seja raiz do polinômios então  $x^2 -(1+a)x + 9 = (x - 3)^2$.

        \begin{align*}
          1+a &= 6 \\
          a & = 5
        \end{align*}
      
      

          \item  Dois indivíduos tentam se comunicar utilizando expressões matemáticas. A transmissão de um número é obtida resolvendo uma expressão matemática. O indivíduo “A” envia a seguinte mensagem 

\begin{equation}
  \left( \frac{x + \sqrt{x^2-4x}}{x - \sqrt{x^2-4x}}  - \frac{x - \sqrt{x^2-4x}}{x + \sqrt{x^2-4x}} \right)\left( \frac{1}{\sqrt{x^2-4x}}\right)
\end{equation}
Você é o indivíduo B e precisa simplificar expressão para determinar que número
dos conjuntos reais a mensagem se refere.

\item Determine a forma fatorada do produto $(x^2 - 14x + 49 )(x^2 + 14x + 49 )$
\item Determine a for-ma simplificada $(x^2 + 14x + 49)( x^2 - 49)/(x^2 - 14x +
  49)$
  \item  A forma simplificada da razão entre os polinômios $x^3 - 8y^3$ e $x^2 -
    4xy + 4y^2$

  \item Seja $C=\{ x \in \mathbb{R} | -2 \leq x \leq 2  \}$, responda os itens a seguir.  

    \begin{enumerate}[a)]
    \item  Qual sua representação através da reta?
    \item  Se $y = \sqrt{-x}$ , represente o conjunto ao qual y pertence.
    \item Se $y = -\sqrt{x}$, represente o conjunto ao qual y pertence.
    \end{enumerate}

Resposta

       \begin{enumerate}[a)]
       \item  Qual sua representação através da reta?


         \begin{figure}
           \centering
           \begin{tikzpicture}
             \draw [->] (-3,0) -- (3,0) node[below] {$x$};
             \draw [red] (-2,0) node[circle,fill=black!90] {} -- (2,0) node[circle,fill=black!90] {};
           \end{tikzpicture}
           \caption{Representação na reta}
         \end{figure}
         
    \item  Se $y = \sqrt{-x}$ , represente o conjunto ao qual y pertence.
    \item Se $y = -\sqrt{x}$, represente o conjunto ao qual y pertence.
    \end{enumerate}
\end{enumerate} % fim das questões
\end{document}