%%% Modelo de Lista de Exercícios para Universidade
%%% Criado por Diogo Roberto R. Freitas (diogo@poli.br)
%%% Livre para alterações
\documentclass[12pt,onepage,a4paper]{memoir}

%% Language and font encodings
\usepackage[english,portuges]{babel}
\usepackage[T1]{fontenc}
\usepackage[utf8]{inputenc}

%% Sets page size and margins
\usepackage[a4paper,top=2.5cm,bottom=2cm,left=2cm,right=2cm,marginparwidth=1.75cm]{geometry}
\setlength\parindent{0cm} % Tamanho da tabulação dos parágrafos

%%% ATENÇÃO!!! %%%
%%% Preencha estes comandos com suas informações
\newcommand{\logo}{\includegraphics[width=0.4\textwidth]{fig/unichristus}} % inclua o arquivo com o logo da instituição
\newcommand{\univ}{Centro Universitário Christus}
\newcommand{\escola}{Sistemas de Informação}
\newcommand{\disc}{Matemática Aplicada}
\newcommand{\auth}{Prof. Daniel Araújo}
\newcommand{\email}{\url{repositoriouni@gmail.br}}
\newcommand{\sitedisc}{\url{--}} % site da disciplina
\newcommand{\cabec}{UNICHRISTUS} % cabeçalho a partir da 2ª página
\newcommand{\tit}{Revisão para NP2}
\newcommand{\tp}[1]{(Tópico: #1)- }
%%%%%%%%%

%% Useful packages
\usepackage{amsmath,amsfonts}
\usepackage{graphicx}
\usepackage[colorinlistoftodos]{todonotes}
\usepackage[colorlinks=true, allcolors=blue]{hyperref}
\linespread{1.25}
\usepackage{graphicx}
\usepackage{nicefrac}
\usepackage[tight]{units}
\usepackage[justification=centering]{caption}
\usepackage{subcaption}
\usepackage{lastpage}
\usepackage{pstricks}
\usepackage{url}% ou hyperref
%\usepackage{breakurl}
\usepackage{multirow}
\usepackage{tabulary}
\usepackage{longtable}
\usepackage{microtype}% improves the spacing between words and letters
\usepackage{booktabs}%  helps you improve the quality of your LaTeX tables
\usepackage{rotfloat}
\usepackage{rotating}

%%% NÃO ALTERAR ESTES COMANDOS 
%%% Configura a primeira página
\makepagestyle{1pagina}
\makeoddhead{1pagina}{
	\logo \\
    \vspace{5pt}
    %\textsf{\univ \\ \escola \\} %%% Caso o logo não tenha o nome da universidade descomente essa linha
    \textsf{Disciplina: \disc \\
	\auth~(\email) \\
    \sitedisc }
    }{}{}
\makeoddfoot{1pagina}{\tiny \cabec}{}{\scriptsize Página \thepage~de \thelastpage}
\makefootrule{1pagina}{\textwidth}{\normalrulethickness}{5pt}
%%% Configura as demais páginas
\makepagestyle{paginacomum}
\makeevenhead{paginacomum}{\textsf{\scriptsize \cabec~-- \auth}}{}{}
\makeevenfoot{paginacomum}{\tiny \cabec}{}{\scriptsize Página \thepage~de \thelastpage}
\makeoddhead{paginacomum}{\textsf{\scriptsize \cabec~-- \auth}}{}{}
\makeoddfoot{paginacomum}{\tiny \cabec}{}{\scriptsize \thepage~de \thelastpage}
\makeheadrule{paginacomum}{\textwidth}{\normalrulethickness}
\makefootrule{paginacomum}{\textwidth}{\normalrulethickness}{5pt}
\pagestyle{paginacomum}
%%%


%%% Início do documento
\begin{document}
\thispagestyle{1pagina}
\vspace*{2.5cm} %%% Caso o cabeçalho cubra estes texto aumente o "vspace"
\fbox{
  \begin{minipage}{\textwidth}
    Instruções: Exercícios de revisão para  NP2
    \end{minipage}
  
}

\vspace{0.5cm}
\textbf{\textsf{\large \tit}} %%% não altere aqui



\begin{enumerate} % início das questões, pode escrever aqui
  % Quest 1
\item  Resolva as equações a seguir:

  \begin{enumerate} [a)]
  \item $18x - 43 = 65$
  \item $23x - 16 = 14 - 17x$
  \item $10y - 5 (1 + y) = 3 (2y - 2) - 20$
  \item $x(x + 4) + x(x + 2) = 2x^{2} + 12 $
  \item $(x - 5)/10 + (1 - 2x)/5 = (3-x)/4$
    \item $4x (x + 6) + x^{2} = 5x^{2}$
  \end{enumerate}

\item Determine um número real "a" para que as expressões $(3a + 6)/ 8 e (2a +
  10)/6$ sejam iguais

  \item Um grupo de 50 pessoas fez um orçamento inicial para organizar uma festa, que seria dividido entre elas em cotas iguais. Verificou-se ao final que, para arcar com todas as despesas, faltavam R\$ 510,00, e que 5 novas pessoas haviam ingressado no grupo. No acerto foi decidido que a despesa total seria dividida em partes iguais pelas 55 pessoas. Quem não havia ainda contribuído pagaria a sua parte, e cada uma das 50 pessoas do grupo inicial deveria contribuir com mais R\$ 7,00.

De acordo com essas informações, qual foi o valor da cota calculada no acerto
final para cada uma das 55 pessoas?

\item As raízes da equação $-x^{2} + 6x = 5$ representam a quantidade de vagas em certo concurso público para os cargos de instalador hidráulico e operador de estação de bombeamento. Sabendo-se que a quantidade de vagas para o cargo de instalador hidráulico foi maior do que a quantidade de vagas para o cargo de operador de estação de bombeamento, quantas são as vagas para o cargo de operador de estação de bombeamento?

\item Identifique os coeficientes de cada equação e obtenha suas raízes:

  \begin{enumerate}[a)]
  \item $5x^2 - 3x - 2 = 0$
  \item $3x^2  + 55 = 0$
  \item $ x^2 - 6x = 0$
    \item $x^2 - 10x + 25 = 0$
    \end{enumerate}

    \item O número -3 é a raíz da equação $x^2 - 7x - 2c = 0$. Nessas condições, determine o valor do coeficiente c.

    \item Considere a função composta

      \[f(x) =  \begin{cases} 
      2  &, x < -4 \\
      x^2 - 4 &, -4\leq x\leq 4 \\
      x-4 &, x>4 
   \end{cases}.
\]
Obtenha o gráfico de $f(x)$ e encontre as raízes.
  
\end{enumerate} % fim das questões
\end{document}
%%% Local Variables:
%%% mode: latex
%%% TeX-master: t
%%% End:
